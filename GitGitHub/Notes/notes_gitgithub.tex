%\listfiles
\documentclass[a4paper,12pt,twoside]{article}

%% packages
\usepackage[french]{babel}
\usepackage[T1]{fontenc}
\usepackage[utf8]{inputenc}
\usepackage{amsmath,amssymb,mathrsfs} 
%\usepackage{import}
\usepackage{float, subfig, caption}
\usepackage{libertine}
\usepackage{fancybox}
\usepackage{graphicx}
\usepackage{xcolor} %/ pour \Definecolor
\usepackage[pdftex,final=true]{hyperref}
%\usepackage[]{hyperref}
\hypersetup{
colorlinks=true,
linkcolor=blue,
urlcolor=magenta, %/ file: adopte également cette couleur
filecolor=cyan, % ne fonctionne qu'avec run:, 
%citecolor=red
}
\urlstyle{same}
\usepackage[french]{minitoc}
\usepackage[acronym,toc,automake]{glossaries} % acronym pour les acronymes et toc pour inclusion des glossary dans la table des matières
\usepackage[left=2cm,right=2cm,top=2cm,bottom=2cm]{geometry}
%\usepackage{minted} % pour code avec coloration dans le texte
\usepackage{url} % Pour avoir de belles url
%% package ticks spécifiquement (pour les schémas) 
%\usepackage{tikz}%
%\usepackage[top=2cm,bottom=2cm]{geometry}
%% paramètres tikz
%\usetikzlibrary{patterns,decorations.pathreplacing,shapes.misc}
%\usetikzlibrary{calc}
%\tikzset{cross/.style={cross out, draw=black, minimum size=2*(#1-\pgflinewidth), inner sep=0pt, outer sep=0pt},
%default radius will be 1pt. 
%cross/.default={5pt}}
% \definecolor{Gray}{gray}{0.9}
% \definecolor{LightCyan}{rgb}{0.88,1,1}
% \newcommand{\RN}[1]{\textup{\uppercase\expandafter{\romannumeral#1}}}    

%% Macros comme variable
%\newcommand{\editorpath}{C:/Program Files/Notepad++/notepad++.exe}
\newcommand{\exemplepath}{../Exercices}
\newcommand{\urlcolor}{magenta}  % couleur des liens https
\newcommand{\filecolor}{cyan} % couleur des liens file: file
\newcommand{\pdfcolor}{red} % couleur des liens file: .pdf
\newcommand{\foldercolor}{orange} % couleur des liens file: Folder
\newcommand{\cscolor}{green!60!black} % couleur des liens file: .cs
\newcommand{\slncolor}{violet} % couleur des liens file: .sln
\newcommand{\keycolor}{purple} % couleur des liens  de type mot clefs
%% Macros commun
\newcommand{\incode}[1]{{\footnotesize\ttfamily #1}} % pour du pseudo-code simple
\newcommand{\rem}[2]{\noindent\underline{Remarque} : \textit{#1}.\\ \indent #2}
\newcommand{\note}[1]{\noindent\underline{Note} : \\ \indent #1}
\newcommand{\defi}[2]{\noindent\underline{Définition} : \textbf{#1}.\\ \indent #2}
\newcommand{\slide}[2]{\textbullet ~ Slide n°#1 : \indent #2}
%% Macro spécifiques à ce fichier (pas forcément) : pour fonctionner il faut que le fichier ai été ouvert diretcement dans le dossier du .tex ...
\newcommand{\fileref}[2]{\hypersetup{urlcolor=\filecolor}\href{file: #1}{#2}\hypersetup{urlcolor=\urlcolor}}
\newcommand{\pdfref}[2]{\hypersetup{urlcolor=\pdfcolor}\href{file: #1.pdf}{#2.pdf}\hypersetup{urlcolor=\urlcolor}}
\newcommand{\csref}[2]{\hypersetup{urlcolor=\cscolor}\href{file:\exemplepath /#1.cs}{#2.cs}\hypersetup{urlcolor=\urlcolor}}
\newcommand{\slnref}[2]{\hypersetup{urlcolor=\slncolor}\href{file:\exemplepath /#1.sln}{#2.sln}\hypersetup{urlcolor=\urlcolor}}
\newcommand{\folderref}[2]{\hypersetup{urlcolor=\foldercolor}\href{file:\exemplepath /#1/.}{#2}\hypersetup{urlcolor=\urlcolor}}
\newcommand{\keyref}[2]{\hypersetup{urlcolor=\keycolor} \href{#1}{\textbf{#2}}\hypersetup{urlcolor=\urlcolor}}
\newcommand{\keyword}[3]{\noindent\underline{Mot clef} : \keyref{#1}{#2}. \\ \indent #3}
%% Macro spécifiques à ce fichier (pas forcément)
\newcommand{\dpat}[7]{
\noindent \underline{Pattern} : \textbf{#1}, \indent (type : \textit{#2}) \\
\underline{Description} : \indent #3 \\
\underline{Exemple} : \indent #4 \\
\noindent \underline{Avantages} : 
\begin{itemize}
 #5
\end{itemize}
\noindent \underline{Inconvénients} : 
\begin{itemize}
 #6 
\end{itemize}
\underline{Liens} : #7 
}

%% Infos document
% Titre
\title{Notes sur le module : \textit{Git et GitHub}.}
% auteur
\author{Armel Pitelet}
% date de création du document
\date{\today}

%% Glossaire, acronymes et index
\makeglossaries

%% Glossaire et acronyme (déclarations)

% DIP : exemple de formalisme
\newglossaryentry{gl-dip}{
name = {DSP},
description = {la description détaillé}
}
\newglossaryentry{ac-dip}{
type = \acronymtype, name = {DSP}, 
description = {Nom simple}, 
first = {Première occurence\glsadd{gl-dip}}, 
see = [Glossaire:]{gl-dip}
}


\begin{document}
%% Organisation globale 
\maketitle
\tableofcontents

%%--------------------------------------------------------------------------------------------------------------------------------
%% Contenue
\section{GitHub}

\note{Il faut faire le readme le plus vite possible.}\\

\note{Il faut faire des \textit{issue} pour chaque tache à réaliser.}\\

\rem{Sur la partie project}{En utilisant le template \textbf{kanban} on a ensuite accès à un tableau type agile (Todo, in progress, done). Une fois dans ce menu on peut également assigné une personne à une tache.}\\

\rem{Sur les pull-request}{La syntax \# issu number peut être utilisé dans le comment d'une pull request pour mentionner la résilution d'une \textit{issue} en particulier.}\\

\defi{Code smells}{Petite typo, code inutile, redondance inutiles, mauvaise visibilité, mauvaise convention de nommage...}\\

\note{\keyref{https://www.sonarqube.org/}{Sonarqube} et \keyref{https://sonarcloud.io/}{Sonarcloud} permettent l'analyse du code : coverage, code smells, duplication de code, vulnérabilité...}\\

\rem{Lors d'une pull request}{Il vaut mieux faire des commentaires pour puis les résoudre.}\\

\rem{Sur le squash}{Permet de compacter les différents commits pour les fusionner en un seul (utile si beaucoup de commits pas très utiles).}\\

\rem{Sur les forks}{On peut faire des \textit{forks} qui vont ouvrir un nouveau repo puis pull request ces \textit{forks} pour les remerge dans le repos principale.}\\

\note{Pendant un \textit{pull request} tt les nouveaux commits de la branche en question apparaitront dans la pull request. Cela permet de modifier le code de la branche pendant l'opération de pull request.}\\

\note{Il est possible de faire des branches à partir de branches, puis de pull request pour merge ces sous branches dans une branche.}\\

\note{\incode{git push --set-upstream origin branch_name} permet de pousser une branche locale sur le github.}\\

\rem{Sur les Actions}{Il faut utiliser un fichier action.yml pour configurer les différentes actions qui doivent être réalisé correctement pour valider un pull-request (compilation du code, test sonar, test unitaire,...). Voir le \keyref{https://github.com/SonarSource/sonarcloud-github-action}{github} de sonar cloud pour commencer ce genre de chose}\\



%%--------------------------------------------------------------------------------------------------------------------------------1
%% Glossaires
\newpage
\printglossary[type=\acronymtype]%\addcontentsline{toc}{chapter}{Acronymes}
%\glsaddall% force l'apparition de tt les entrées du glossaire
\glsaddallunused % meme chose que addall mais ne force pas la numéroattion dans la liste d'acronyme
%\printunsrtglossary[type=main]
\printglossary[type = main,nonumberlist]%\addcontentsline{toc}{chapter}{Glossaire}

\end{document}
\end{document}
