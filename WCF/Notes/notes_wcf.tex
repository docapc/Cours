%\listfiles
\documentclass[a4paper,12pt,twoside]{article}

%% packages
\usepackage[french]{babel}
\usepackage[T1]{fontenc}
\usepackage[utf8]{inputenc}
\usepackage{amsmath,amssymb,mathrsfs} 
%\usepackage{import}
\usepackage{float, subfig, caption}
\usepackage{libertine}
\usepackage{fancybox}
\usepackage{graphicx}
\usepackage{xcolor} %/ pour \Definecolor
\usepackage[pdftex,final=true]{hyperref}
%\usepackage[]{hyperref}
\hypersetup{
colorlinks=true,
linkcolor=blue,
urlcolor=magenta, %/ file: adopte également cette couleur
filecolor=cyan, % ne fonctionne qu'avec run:, 
%citecolor=red
}
\urlstyle{same}
\usepackage[french]{minitoc}
\usepackage[acronym,toc,automake]{glossaries} % acronym pour les acronymes et toc pour inclusion des glossary dans la table des matières
\usepackage[left=2cm,right=2cm,top=2cm,bottom=2cm]{geometry}
%\usepackage{minted} % pour code avec coloration dans le texte
\usepackage{url} % Pour avoir de belles url
%% package ticks spécifiquement (pour les schémas) 
%\usepackage{tikz}%
%\usepackage[top=2cm,bottom=2cm]{geometry}
%% paramètres tikz
%\usetikzlibrary{patterns,decorations.pathreplacing,shapes.misc}
%\usetikzlibrary{calc}
%\tikzset{cross/.style={cross out, draw=black, minimum size=2*(#1-\pgflinewidth), inner sep=0pt, outer sep=0pt},
%default radius will be 1pt. 
%cross/.default={5pt}}
% \definecolor{Gray}{gray}{0.9}
% \definecolor{LightCyan}{rgb}{0.88,1,1}
% \newcommand{\RN}[1]{\textup{\uppercase\expandafter{\romannumeral#1}}}    

%% Macros comme variable
%\newcommand{\editorpath}{C:/Program Files/Notepad++/notepad++.exe}
\newcommand{\exemplepath}{../Exercices}
\newcommand{\urlcolor}{magenta}  % couleur des liens https
\newcommand{\filecolor}{cyan} % couleur des liens file: file
\newcommand{\pdfcolor}{red} % couleur des liens file: .pdf
\newcommand{\foldercolor}{orange} % couleur des liens file: Folder
\newcommand{\cscolor}{green!60!black} % couleur des liens file: .cs
\newcommand{\slncolor}{violet} % couleur des liens file: .sln
\newcommand{\keycolor}{purple} % couleur des liens  de type mot clefs
%% Macros commun
\newcommand{\incode}[1]{{\footnotesize\ttfamily #1}} % pour du pseudo-code simple
\newcommand{\rem}[2]{\noindent\underline{Remarque} : \textit{#1}.\\ \indent #2}
\newcommand{\note}[1]{\noindent\underline{Note} : \\ \indent #1}
\newcommand{\defi}[2]{\noindent\underline{Définition} : \textbf{#1}.\\ \indent #2}
\newcommand{\slide}[2]{\textbullet ~ Slide n°#1 : \indent #2}
%% Macro spécifiques à ce fichier (pas forcément) : pour fonctionner il faut que le fichier ai été ouvert diretcement dans le dossier du .tex ...
\newcommand{\fileref}[2]{\hypersetup{urlcolor=\filecolor}\href{file: #1}{#2}\hypersetup{urlcolor=\urlcolor}}
\newcommand{\pdfref}[2]{\hypersetup{urlcolor=\pdfcolor}\href{file: #1.pdf}{#2.pdf}\hypersetup{urlcolor=\urlcolor}}
\newcommand{\csref}[2]{\hypersetup{urlcolor=\cscolor}\href{file:\exemplepath /#1.cs}{#2.cs}\hypersetup{urlcolor=\urlcolor}}
\newcommand{\slnref}[2]{\hypersetup{urlcolor=\slncolor}\href{file:\exemplepath /#1.sln}{#2.sln}\hypersetup{urlcolor=\urlcolor}}
\newcommand{\folderref}[2]{\hypersetup{urlcolor=\foldercolor}\href{file:\exemplepath /#1/.}{#2}\hypersetup{urlcolor=\urlcolor}}
\newcommand{\keyref}[2]{\hypersetup{urlcolor=\keycolor} \href{#1}{\textbf{#2}}\hypersetup{urlcolor=\urlcolor}}
\newcommand{\keyword}[3]{\noindent\underline{Mot clef} : \keyref{#1}{#2}. \\ \indent #3}
%% Macro spécifiques à ce fichier (pas forcément)
\newcommand{\dpat}[7]{
\noindent \underline{Pattern} : \textbf{#1}, \indent (type : \textit{#2}) \\
\underline{Description} : \indent #3 \\
\underline{Exemple} : \indent #4 \\
\noindent \underline{Avantages} : 
\begin{itemize}
 #5
\end{itemize}
\noindent \underline{Inconvénients} : 
\begin{itemize}
 #6 
\end{itemize}
\underline{Liens} : #7 
}

%% Infos document
% Titre
\title{Notes sur le module : \textit{WPF}.}
% auteur
\author{Armel Pitelet}
% date de création du document
\date{\today}

%% Glossaire, acronymes et index
\makeglossaries

%% Glossaire et acronyme (déclarations)

% DIP : exemple de formalisme
\newglossaryentry{gl-dip}{
name = {DSP},
description = {la description détaillé}
}
\newglossaryentry{ac-dip}{
type = \acronymtype, name = {DSP}, 
description = {Nom simple}, 
first = {Première occurence\glsadd{gl-dip}}, 
see = [Glossaire:]{gl-dip}
}


\begin{document}
%% Organisation globale 
\maketitle
\tableofcontents

%%--------------------------------------------------------------------------------------------------------------------------------
%% Contenue

\section{Sur WCF}

\defi{WCF}{Windows Communication Fundation \url{https://fr.wikipedia.org/wiki/Windows_Communication_Foundation}. Actuellement remplacé par ASPNet Core Web API. WCF est Legacy, principalement basé sur OData. Utilisé pour mettre en place des services sur un serveur.}\\

\rem{ASP Net Web Api}{Principalement basé sur Rest \url{https://fr.wikipedia.org/wiki/Representational_state_transfer}.}\\

\rem{Format JSON}{Format descriptif très utilisé dans les appels entre client et serveur. Particularité, accolade ouvrante et fermante encadrant un format clef-valeur pour décrire une donnée. Voir \url{https://jsonplaceholder.typicode.com/} pour des exemple. Le JSON ne porte rien de plus que des clefs-valeurs (pas d'information complémentaire ni de vaidation des données possible [donnée en elle même ou bien des conneries comme l'encodage]). Ce souci est en général réglé par la doc.}\\

\rem{Format XML}{Extensible Markup Language. Format précédent JSON, beaucoup plus descriptif et avec méchanisme de validtion mais beaucoup plus lourd. Voir le format XAML basé sur XML (language balise quoi qu'il arrive). Est interessant car permet de faire de la validation nativement car inclue dans le formalisme (appelé \textit{xsd} [pour XML Schema Definition]). Voir égalemet \keyref{https://fr.wikipedia.org/wiki/GRPC}{gRPC}}\\

\note{WCF est utilisable avec du JSON (via hack...) mais est conçue pour du http, et donc du XML. C'est un composant de communication complet et pas un simple language (mais Legacy donc non recommandé, contrairement à WML qui est juste moins utilisé mais toujours d'actualité en fonction des besoins).}\\

\rem{XML Schéma}{ Voir \url{https://fr.wikipedia.org/wiki/XML_Schema} : permet la validation des données via des schémas que les données doivent respecter.}\\

\rem{Type d'application en WCF}{On sera interessé par WCF Service application (hosted in IIS). Attention WCF fonctionne avec .Net FrameWork et ne sera jamais porté sur .Net Core!!!}\\

\rem{IIS Express}{ Service qui permet d'utiliser un serveur Web.}\\

WCF tourne sur un serveur Web et un projet WCF est forcément accompagné d'un Web.config.  Ce fichier est accompagné d'une interface qui définit les méthodes que le service va exposer, et d'un fichier .svc qui définit l'implémentation du service.\\

\rem{local host}{est un alias de l'ip de la machine en cours (local). localhost:62334/Service1.svc -> le nombre ici désigne le port utilisé pour accéder au service qui tourne sur \textit{localhost}.}\\

WCF utilise des adresses, et des protocoles (comme http, https) pour créer un \textit{client}.\\

\note{Le WCF test client sert à tester le Web Client sans passer par une interface (graphique ou commande). Par défaut Visual Studio (2019 au moins) lance le test client pendant un run.}\\

\rem{Sur les attributs}{ServiceContract est une propriété qui déclare une interface utilisé pour un Web Service WCF. OperationContract rend la méthode décoré avec disponible pour le web service WCF.}\\

\note{Tout ce qui est spécifique à WCF est dans l'interface (IService) et pas dans l'implémentation du Service en lui même (Service.svc.cs).}\\

\note{De base un service contract (l'interface) déclare uniquement des méthodes et pas de propriétés.}\\

\note{Il vaut mieux ne pas créer l'interface avant de créer le fichier .svc (pas le svc.cs mais bien ls svc). De base le svc n'est pas visible il faut clik droit -> \textit{view markup} pour le lire.}\\

\note{La bonne pratique est de faire un service par domaine d'application.}\\

\rem{service statefull et stateless}{Fonction statefull -> seul le premier serveur qui répond récupère l'état de navigation dans le service. Statefull force le client à suivre l'état. Statefull = garder un état de ce qui se passe coté front dans le coté back.}

\rem{Attribut DataContract et DataMember}{Tt objet qui n'est pas un built in doir porter l'attribut DataContract et toutes méthodes ou propriétés associés doit porter l'attribut DataMember.}\\

\note{WCF utilise ensuite un format xmlsoap (avec enveloppe -> avec header) pour décrire et transférer l'objet.}\\

\rem{Pour appeler WCF depuis WPF}{click droit sur le projet WPF/Connected Services -> add -> service référence}\\

\note{dans ce contexte les deux parties WCF et WPF ne partagent pas les donnée en dehors des appels de méthodes.}\\

\note{Il ne faut jamais faire un add reference d'une appli web (serveur WCF) dans l'autre partie de l'appli (WPF). (sauf via le add service référence.)}\\

\note{Si l'on rajoute une méthode coté serveur (que l'on build), il faut mettre à jour le ConnectedService coté client (click droit -> Update Service Reference).}\\

\note{pour connaitre que le nom que le client va prendre il faut lancer le service (wcf) et cliquer sur le fichier svc}


\section{Divers}

\rem{Sur les branches git}{Voir Git Flow (répond à la problématique de la maintenance d'une version simultanée au développement d'une nouvelle version) et GitHub Flow (plus simple car pas de branche dévelop) pour des schémas plus simple que ceux vues en Outils du développement.}\\

\rem{Encodage Windows}{Windows et donc .Net est en UTF-16!!! Attention peut poser des problème avec les système Unix qui sont restés en UTF-8.}\\

\defi{CRLF et LF}{LF : chaque fin de ligne (aussi appelé \textit{retour chariot}), \textbackslash r (un seul élément), CRLF : deux éléments de fin de ligne \textbackslash r \textbackslash n. Peut poser des problèmes de EOF (End Of Lign).}\\

\note{\keyref{https://docs.microsoft.com/en-us/sysinternals/downloads/zoomit}{zoomit} permet de zoomer sur un écran (outils windows).}\\

\note{Lorsque l'on a plusieurs projet dans une solution il faut clik droit sur le projet et sélectionner \textit{set as startup}.}

\note{Lorsque l'on a du WCF et du WPF dans la même solution on peut click-droit sur start ProjectS et l'on peut ensuite sélectionner les lancement et leur ordre (d'abord le serveur [WCF] puis le client [WPF]).}\\

\note{Ctrl+R,R permet de renommer rapidemment dans l'ensemble d'un projet /solution.}\\

\note{rappel namespace : nom\_societe.nom\_projet.nom\_application}

\defi{DTO}{Data transfert Object : object partagé entre plusieurs projets. Quand on fait un dto il vaut mieux garder les objet les plus bêtes possible (propriété uniquement avec getter et setter en public).}\\

\note{Lors des transfert WCF on ne peut envoyer que des données (traduite en texte dans les tuyaux) et pas d'objet complexe donc sans méthode, attribut ou constructeurs.}\\

\rem{Sur le property Required}{Une property required ne peut pas etre nulle ou non définie (obligatoire de définir un get et un set).}\\

%%--------------------------------------------------------------------------------------------------------------------------------1
%% Glossaires
\newpage
\printglossary[type=\acronymtype]%\addcontentsline{toc}{chapter}{Acronymes}
%\glsaddall% force l'apparition de tt les entrées du glossaire
\glsaddallunused % meme chose que addall mais ne force pas la numéroattion dans la liste d'acronyme
%\printunsrtglossary[type=main]
\printglossary[type = main,nonumberlist]%\addcontentsline{toc}{chapter}{Glossaire}

\end{document}
\end{document}
