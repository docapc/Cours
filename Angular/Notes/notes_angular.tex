%\listfiles
\documentclass[a4paper,12pt,twoside]{article}

%% packages
\usepackage[french]{babel}
\usepackage[T1]{fontenc}
\usepackage[utf8]{inputenc}
\usepackage{amsmath,amssymb,mathrsfs} 
%\usepackage{import}
\usepackage{float, subfig, caption}
\usepackage{libertine}
\usepackage{fancybox}
\usepackage{graphicx}
\usepackage{xcolor} %/ pour \Definecolor
\usepackage[pdftex,final=true]{hyperref}
%\usepackage[]{hyperref}
\hypersetup{
colorlinks=true,
linkcolor=blue,
urlcolor=magenta, %/ file: adopte également cette couleur
filecolor=cyan, % ne fonctionne qu'avec run:, 
%citecolor=red
}
\urlstyle{same}
\usepackage[french]{minitoc}
\usepackage[acronym,toc,automake]{glossaries} % acronym pour les acronymes et toc pour inclusion des glossary dans la table des matières
\usepackage[left=2cm,right=2cm,top=2cm,bottom=2cm]{geometry}
%\usepackage{minted} % pour code avec coloration dans le texte
\usepackage{url} % Pour avoir de belles url
%% package ticks spécifiquement (pour les schémas) 
%\usepackage{tikz}%
%\usepackage[top=2cm,bottom=2cm]{geometry}
%% paramètres tikz
%\usetikzlibrary{patterns,decorations.pathreplacing,shapes.misc}
%\usetikzlibrary{calc}
%\tikzset{cross/.style={cross out, draw=black, minimum size=2*(#1-\pgflinewidth), inner sep=0pt, outer sep=0pt},
%default radius will be 1pt. 
%cross/.default={5pt}}
% \definecolor{Gray}{gray}{0.9}
% \definecolor{LightCyan}{rgb}{0.88,1,1}
% \newcommand{\RN}[1]{\textup{\uppercase\expandafter{\romannumeral#1}}}    

%% Macros comme variable
%\newcommand{\editorpath}{C:/Program Files/Notepad++/notepad++.exe}
\newcommand{\exemplepath}{../Exercices}
\newcommand{\urlcolor}{magenta}  % couleur des liens https
\newcommand{\filecolor}{cyan} % couleur des liens file: file
\newcommand{\pdfcolor}{red} % couleur des liens file: .pdf
\newcommand{\foldercolor}{orange} % couleur des liens file: Folder
\newcommand{\cscolor}{green!60!black} % couleur des liens file: .cs
\newcommand{\slncolor}{violet} % couleur des liens file: .sln
\newcommand{\keycolor}{purple} % couleur des liens  de type mot clefs
%% Macros commun
\newcommand{\incode}[1]{{\footnotesize\ttfamily #1}} % pour du pseudo-code simple
\newcommand{\rem}[2]{\noindent\underline{Remarque} : \textit{#1}.\\ \indent #2}
\newcommand{\note}[1]{\noindent\underline{Note} : \\ \indent #1}
\newcommand{\defi}[2]{\noindent\underline{Définition} : \textbf{#1}.\\ \indent #2}
\newcommand{\slide}[2]{\textbullet ~ Slide n°#1 : \indent #2}
%% Macro spécifiques à ce fichier (pas forcément) : pour fonctionner il faut que le fichier ai été ouvert diretcement dans le dossier du .tex ...
\newcommand{\fileref}[2]{\hypersetup{urlcolor=\filecolor}\href{file: #1}{#2}\hypersetup{urlcolor=\urlcolor}}
\newcommand{\pdfref}[2]{\hypersetup{urlcolor=\pdfcolor}\href{file: #1.pdf}{#2.pdf}\hypersetup{urlcolor=\urlcolor}}
\newcommand{\csref}[2]{\hypersetup{urlcolor=\cscolor}\href{file:\exemplepath /#1.cs}{#2.cs}\hypersetup{urlcolor=\urlcolor}}
\newcommand{\slnref}[2]{\hypersetup{urlcolor=\slncolor}\href{file:\exemplepath /#1.sln}{#2.sln}\hypersetup{urlcolor=\urlcolor}}
\newcommand{\folderref}[2]{\hypersetup{urlcolor=\foldercolor}\href{file:\exemplepath /#1/.}{#2}\hypersetup{urlcolor=\urlcolor}}
\newcommand{\keyref}[2]{\hypersetup{urlcolor=\keycolor} \href{#1}{\textbf{#2}}\hypersetup{urlcolor=\urlcolor}}
\newcommand{\keyword}[3]{\noindent\underline{Mot clef} : \keyref{#1}{#2}. \\ \indent #3}
%% Macro spécifiques à ce fichier (pas forcément)
\newcommand{\dpat}[7]{
\noindent \underline{Pattern} : \textbf{#1}, \indent (type : \textit{#2}) \\
\underline{Description} : \indent #3 \\
\underline{Exemple} : \indent #4 \\
\noindent \underline{Avantages} : 
\begin{itemize}
 #5
\end{itemize}
\noindent \underline{Inconvénients} : 
\begin{itemize}
 #6 
\end{itemize}
\underline{Liens} : #7 
}

%% Infos document
% Titre
\title{Notes sur le module : \textit{Angular.}}
% auteur
\author{Armel Pitelet}
% date de création du document
\date{\today}

%% Glossaire, acronymes et index
\makeglossaries

%% Glossaire et acronyme (déclarations)

% DIP : exemple de formalisme
\newglossaryentry{gl-dip}{
name = {DSP},
description = {la description détaillé}
}
\newglossaryentry{ac-dip}{
type = \acronymtype, name = {DSP}, 
description = {Nom simple}, 
first = {Première occurence\glsadd{gl-dip}}, 
see = [Glossaire:]{gl-dip}
}


\begin{document}
%% Organisation globale 
\maketitle
\tableofcontents

%%--------------------------------------------------------------------------------------------------------------------------------
Lien de la doc Angular \url{https://angular.io/docs}. \\
Lien de la doc Mozilla \url{https://developer.mozilla.org/en-US/} (html, css, javascript, typescript).\\

\section{Bases}

\defi{Angular}{framework (cadre de travail) permettant de faire des applications bureau mais surtout web. Historiquement c'était Angularjs et maintenant Angular 13. Maintenue par google, il est réttocompatible et sécurisé. C'est un framework principalement front-end.}\\

\rem{Sur les commandes angular}{Ces commandes commencent par \incode{ng} (dans un terminal).}\\

\rem{Any desk}{Un genre de team viewer (mais en ligne).}\\

\rem{Pour créer un projet}{\incode{neg new NOM\_PROJET}. puis (\incode{n} : pas dans la dernière version), \incode{y} (routing) et sélectionner un language de style.}\\

\rem{Sur html et css}{html sert à faire du formatage de page et css (fiche de style en cascade) à faire du formatage de style. Angular utilise les deux pour faire des pages web.}\\

\defi{SCSS}{CSS avec une écriture moins lourde que css. Attention dans un navigateur on ne peut voir que du css. SCSS doit être compiler vers CSS avec Angular pour être interprété par un naviguateur.}\\

\note{Pour l'édition de code il faut ensuite passer par Visual Studio Code ou bien PhpStorm (payant mais un mois de licence gratuite...)}\\

\rem{Pour excécuter des scripts (angular mais pas que)}{\incode{set-executionpolicy unrestricted} dasn un powershell en mode administrateur (2 conditions obligatoires). Nécessaire pour travailler avec Angular.}\\

\note{Les projets Angular sont super Fat en terme de place sur le disque.}\\

\note{Il faut ouvrir le dossier de projet entier (dans VSCode) pour que le serveur puisse fonctionner par la suite.}\\

\rem{Sur l'arborescence d'un projet}{
\begin{itemize}
\item \incode{tsconfig.json} : config pour la compilation du Typescript vers du javascript (javascript est par contre un Typescript valide). Ce fichier est nécessairepour faire du typescript (et ce n'est pas propre à Angular).
\item \incode{package.json} : fichier de dépendance du projet. Si l'on utilise ce fichier la première chose à faire est un \incode{npm install} : pour installer les dépendances d'un projet (va charger le package json dans le nodemodule).
\item Dans le dossier \incode{app} : \incode{app.component.xxx} permet de partitionner l'application en component. Un compenent à un rôle précis $\to$ il gère un seule et unique action. Chaque componenet doit être l eplus indépendant possible. Exemple : le html gère le rendu d'affichage, le css les style, spec.ts sert aux test et le ts est le \textit{controlleur} en lui même. Chaque nouveau component aura 4 fichier mon\_component\_component associés.
\item Dans le dossier \incode{app} : \incode{app.module.ts} permet d'indiquer les librairies nécessaires à l'application (unqiuement ceux inclues dans l'application et pas tt les modules du node module).
\item Dans le dossier \incode{app} : \incode{app-routing.modules.ts} : pour gérer les routes associées aux différents components. Ce fichier lie chaque component à un nom.
\item Dans le dossier \incode{index.html} : est la page principale du projet (dans laquel sera importer le projet après compilation).
\item Dans le dossier \incode{styles.scss} : fichier scss mère qui sera utilisé partout ailleurs dans le projet. Par contre le scss des components est généré en premier (ce qui est important car l'ordre de redéfinition des classe est important.).\\
\end{itemize}
}

\rem{En css}{L'ordre des importation est très important. On peut modifier un import précédent avec un import suivant.}\\

\defi{Bootstrap}{Une bibliothèque CSS qui fournit un ensemble d'objets.}\\

\note{Pour récupérer des données en Angular il faut forcément passer par une API, il n'est pas possible d'interroger directement une base de donnée.}\\

\rem{Sur la compilation}{Une fois la compilation et le serveur lancé (\incode{ng serve --open}) il n'est pas nécessaire de l'arréter. En JavaScript le code est rechargé en permanence.}\\

\rem{Sur l'objet en Angular}{En Angular (et typescript) tt est objet. On en revient donc au un fichier, une classe. Au niveau de la syntax d'une variable nom: type = valeur.}\\

\keyword{https://developer.mozilla.org/fr/docs/Web/JavaScript/Reference/Statements/let}{let}{permet de déclarer une varibale locale modifiable (javascript). Le typage n'est pas obligatoire mais est en théorie obligatoire.}\\

\keyword{https://developer.mozilla.org/fr/docs/Web/JavaScript/Reference/Statements/const}{const}{permet de déclarer une varibale locale non-modifiable (javascript). Le typage n'est pas obligatoire mais est en théorie obligatoire.}\\

\keyword{}{any}{à la place d'un type, permet de déclarer n'importe quel type. A oublier le plus vite possible (car l'intrêt du typescript est de typé au maximum le code.)}\\

\note{A partir du moment ou un attribut ou une fonction (dans le component.ts) est public (visibilité par défaut) il est possible de l'utiliser dans la partie html du component.}\\

\keyword{}{export}{devant la déclaration d'une classe. permet de dire que la classe va pouvoir etre utilisée en dehors du fichier de déclaration. Sans \incode{export} le componet ne peut être utilisé nulle part.}\\

\rem{ng-template}{est un template qui est masqué sur une page tant qu'il n'a pas été appelé. Peut servir avec un *ngIf. Voir Cours et exemple}\\

\keyword{https://developer.mozilla.org/fr/}{Doc html/css}{regarder ce site en cas de questions sur les languages balises à la con.}\\

\rem{Pour créer un nouveau component}{ng generate component component\_name (ou ng g c component\_name si on passe par les alias).}\\

\keyword{}{implements}{permet d'implémenter une interface.}\\

\keyword{}{extends}{permet de déclarer un héritage.}\\

\rem{Sur l'interface OnInit}{fait partie du cycle de vie des objets de type component. OnInit indique un traitement appliqués après l'initialisation d'un objet (constructeur, injection de dépendance). Le OnInit est fait avant que la vue soit générer. En générale on fait dans le OnInit les requêtes à l'api, gérer les paramètres de root.}\\

\keyref{https://www.cuelogic.com/blog/angular-lifecycle}{Sur le cycle de vie des objets angular.}\\

\rem{Sur OnDestroy}{Est appelé lorsque la fenêtre associée au component disparait. }\\

\rem{pour installer une dépendance/librairie}{\incode{npm install un\_truc} ou \incode{npm i un\_truc} si l'on passe par les alias. Met à jour le package.json.}\\

\rem{Sur le SCSS}{On peut définir des variables en SCSS en préfixant le nom d'un d'un dollar : \$variable . \& rappel l'élément courant (parent plutôt, un peu l'équivalent du this).}\\

\rem{console.log(string)}{permet d'afficher une string dans la console log. Si on lui passe un objet la console affichera tt les attributs de la console}

\rem{[ngStyle]="\{'x' ; source\}"}{ est une directive Angular qui permet de chercher une propriété dans une classe source pour l'affecter ici un Style html. Permet de binder à la propriété Style la valeur x extraite de source. Pour plus de détails voir : \url{https://angular.io/api/common/NgStyle}.}\\

\keyword{}{super}{permet de rappeler le constructeur de la classe mère. Dans le constructeur c'est la première instruction qu'il faut appeler. En dehors on peut l'appler à tt moment.}\\

\rem{rem}{unité de mesure relative qui se base sur la police du body (environnement le plus extérieur dans une page). Pour plus de détail sur les tailles : \url{https://developer.mozilla.org/en-US/docs/Learn/CSS/Building_blocks/Values_and_units}}\\

\rem{Sur le padding}{Il s'agit de l'écart entre entre le placement d'objet dans unvironnement et les bords de l'environnement. Il faut regarder sur une page (dans les outils de debug) pour visualiser sa.}\\

\rem{sur le *ngFor}{Ce dernier peut fonctionner sur des container entier tant qu'ils sont contenue à l'intérieur de ces derniers.}\\

\rem{Sur le dossier \incode{asset}}{Dans ce dossier ; tt les ressources nécessaires pour le site (images, musique, video, gif...).}\\

\rem{Pour afficher une image}{balise <img src = "chemin">. Attention la balise ne se ferme pas comme les autres. On peut également utiliser le Binding d'Angular [img]=hero.image au lieu de src = "\{\{hero.image\}\}".}\\

\rem{Pour régler les problèmes de taille d'image}{Il faut faire un environnement <div class="container-image"> ou container-image qui est définit dans le .scss avec un enfant de type balise image (déclaré > img\{width = x\%, height = y\%\} dans le scss. Se lit pour tt balises image contenues dans un container-image alors on appliques ces règles). Attention ce chevron (\textit{chevron selector}) ne fonctionne que pour un enfant direct (pas pour un enfant d'un enfant), au niveau du html.}\\

\rem{sur [class.une\_classe] = condition}{Signifie que l'on applique la classe css si la condition est vrai.}\\

\rem{Sur le partage de classe entre component}{Même avec l'export il n'est pas possible de directement appeler une classe d'un component dans un autre component. Par contre il est possible de faire des classes communes à tt les components et donc utilisables par tous.}\\

\rem{Sur la directive [hide]="condition"}{Permet de masquer la div dans lequel il est contenue si la condition à l'intérieur est réalisée.}\\

\keyword{https://developer.mozilla.org/en-US/docs/Web/CSS/CSS_Selectors}{Sur les selector}{Les sélecteurs permettent de définir les éléments sur lesquels appliquer un ensemble de règles css.}\\

\rem{Sur le fichier app-touting.module.ts}{On donne à la variables Routes un objet json (d'ou les accolades). Pour chaques routes on doit définir le \incode{path :'chemin\_du\_path', component:nom\_du\_component}.}\\

\rem{Sur la balise <router-outlet></router-outlet>}{Indique à partir de quel endroit on intègre un component par rapport à la route que l'on saisie dans l'url. C'est un dicateur qui indique que l'on veut afficher des components à partir d'une route. Avant l'appel à <router-outlet></router-outlet> on peut déclarer une balise header avec une balise <a class $=$"btn" routerLink ="/chemin\_du\_path">Nom\_du\_bouton</a> (qui déclare un lien sur la route routerLink qui doit être déclarer dans le ficheir app-routing.modules.ts). On aurait aussi pu faire [routerLink]="nom\_d\_une\_variable".}\\

\rem{<footer></footer>}{déclare un footer (footnote) sur une page.}

\note{Dans un constructeur on peut mettre une visibilité aux arguments! Si l'argument n'existe pas encore comme attribut il sera crée à la volé par le constructeur.}\\

\rem{console.log(une string)}{Permet de faire remonter des messages dans la console (dans le naviguateur) lors de l'éxécution de.}\\

\rem{Sur la notion de filtre et de pipe}{le | permet de déclarer des \keyref{https://angular.io/guide/pipes}{pipes}, qui permettent d'appliquer des filtres particulier pour de la mise en forme.}\\

\note{En générale un componenet est lié à une route.}\\

\rem{Sur l'appel d'un component}{On peut appeler un component directement par sa route bien via son sélecteur. Exemple : \incode{<component\_name></component\_name>}.}\\

\rem{Sur la balise title}{La balise \incode{<title>MonTitre</title>} de changer le nom afficher dans le naviguateur au niveau de la page. Il faut en définir une par component. Cela ne fonctionne pas. Il faut mieux passer par un service (ici setTitle) à injecter dans chaque component dans le OnInit.}\\

\rem{Sur la notion de service}{Il s'agit d'une classe réutilisable au besoin. pour générer un service : \incode{ng g s} (pour \incode{ng generate service}). Un dossier service à la racine de \incode{src} est une bonne idée.}\\

\note{/** puis entrez permet de générer de la doc par défaut au dessus d'une méthode (avec la liste des paramètres d'entrées et de sortie @param et @return).}\\

\note{le décorateur \incode{@Injectable} permet de dire à Angular qu'il peut instancier lui même ce service au besoin.}\\

\rem{Sur la différence entre $==$ et $===$}{On passe en générale par le triple égal pour vérifier lanotion de strict différence (peut jouer si l'on compare deux type différent comme un \incode{string} et un \incode{number}).}

\section{Sur la connection à une API}

\rem{Imports nécessaires}{dans le modules.cs : HttpClientModule.}\\

\note{Il est préférable de service dédié à la gestion des requêtes http.}\\

\rem{Sur le type Observable}{Un Observable est un type qui représente une donnée qui arrive mais on ne sait pas quand exactement. C'est une variable asynchrone.}\\

\rem{Sur les interfaces en TypeScript}{Elle peuvent servir comme les interface classiques ou bien comme un support permettant de tapper directement du json! (dans un objet non instanciable). Pour faire de l'affichage on peut donc se contenter d'une interface au lieu de faire une classe pleine et entière (constructeur, getter, setter, ...).}\\

\note{lorsque l'on récupère les données au format json depuis l'api, il faut que les noms renvoyées par l'api \textit{collent} avec ceux de nos objets/ou interfaces les récupérant.}

\note{Dans le component qui appel le service on s'abonne ensuite à l'objet correspondant à la requete. C'est possible car http client renvoie forcément un observable. Il faut passer une fonction fléchée à subscribe (un peu comme un lambda)}\\

\rem{Sur les fonction fléchées en TypeScript}{On peut utiliser des lambda à la c\# au format \incode{var => var = qqchose\{\})}.}\\

\note{Comme on ne peut pas instancier une interface. Pour l'initialiser à une valeur valide (mm si vide ou null) on peut lui rajouter \incode{|undefinded} après le type.}\\

\rem{Sur le raisonnement en colonne de bootstrap}{Bootstrap peut fonctionner en colonnes (col) allant de 1 à 12. col-4 correspond à 1 tier de la largeur totale ($\frac{4}{12}$), col-3 à un quart, col-6 à la moitié,... On peut aussi utiliser offset qui utilise le même système numéraire que col. Tt sa pour gérer des tailles ou espacements entre éléments. En plus : my-3 correspond à une marge dans le même système le long de l'axe y. mt marge en haut. Le total des colonnes sur une seul ligne doit au final faire 12.}\\

\rem{Un peu plus de Bootstrap}{Bootstrap a des alias. Exemple col-3 peut être remplacé par col-md-3, ou col-sm-2 (voir remarque au dessus). A ce moment la le chiffre est un maximum et l'alias md, sm (y'en a d'autre) gère la partie responsive de la page (l'ajustement auto en fonction de la taille total de la fenêtre.) Voir : \url{https://getbootstrap.com/docs/4.0/layout/grid/} pour plus de détails. On doit par contre enchainer ces indicateurs pour gérer tt les types d'écrans (de vue dans le navigateur) possibles.}\\

\rem{sur le \incode{sort}}{sur une liste (déclaré comme \incode{variable[]}) on peut faire \incode{variable.sort(fonction\_decomparaison)}. Voir dans le code car vraiment contre intuitif.}\\

\rem{Sur le passage de paramètres à des components}{Lorsque l'on écrit le path dans la \incode{approuting.cs} on peut faire \incode{path $=$ mon\_path$+$'/:name'} pour pouvoir ensuite injecter un nom déterminé à la place du paramètre \incode{name} donné dans le path. Dans le component on peut injecter dans le constructeur un objet de type \incode{ActivatedRoute} qui est l'objet qui va permettre de récupérer le nom (via un \incode{subscribe}sur la réception des paramètres). }\\

\note{Pour faire des test via \incode{consol.log} il faut le faire dans les \incode{subscribe}. Cela vient du fait qu'en dehors du \incode{subscribe} la donnée n'est pas récupérée tant qu'elle n'est pas explicitement demandée (le \incode{console.log} en lui même ne suffit pas pour être considéré comme une demande).}\\

\note{Si il y a un tiret - dans le nom d'une variable dans l'api, il faut \textit{protéger} ce nom dans la déclaration de variable dans notre interface (au sens container json) avec des guillemets simples ''.}\\

\rem{Sur le routerLink}{On est pas obligé d'être dans un environnement <a/> pour utiliser cette propriété. (voir la manière dont les pokemons sont atteint via leurs listes).}\\

\rem{Sur \incode{subscribe}}{Le \incode{subscribe} sert un peu comme un gestionnaire de requête async. Sa à l'air différent du subscribe du pattern \textit{Observer}. $\to$ \textbf{A creuser}.}\

\rem{Sur le \incode{?}}{Permet de déclare un objet (une instance de classe) comme probablement indéfinie. C'est un tricks html.}

\rem{Pour transférer un objet d'un component à un autre}{Il faut décorer l'attribut que l'on veut récupérer avec l'attribut  \incode{@input} (et ne pas oublier d'\incode{import} le input dans la liste de package au début du fichier .ts).}\\

\rem{Sur les sélecteurs et le transfert d'objet}{On peut avec sa appeler un objet OA provenant d'un component CB, dans le component CA pour ensuite appeler le component CA dans le component CB $\to$ voir le yahtzee/hero-detail. C'est un exemple alambiqué mais en gros on peut transférer les objet et appeler les components dans d'autres components.}\\




%%--------------------------------------------------------------------------------------------------------------------------------1
%% Glossaires
% \newpage
% \printglossary[type=\acronymtype]%\addcontentsline{toc}{chapter}{Acronymes}
% %\glsaddall% force l'apparition de tt les entrées du glossaire
% \glsaddallunused % meme chose que addall mais ne force pas la numéroattion dans la liste d'acronyme
% %\printunsrtglossary[type=main]
% \printglossary[type = main,nonumberlist]%\addcontentsline{toc}{chapter}{Glossaire}

\end{document}
\end{document}
